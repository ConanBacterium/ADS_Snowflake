\documentclass{article}
\usepackage{graphicx}
\usepackage[margin=1in]{geometry}
\usepackage{tabularx}
\usepackage{booktabs}
\usepackage{makecell}
\usepackage{adjustbox}
\usepackage{placeins}
\usepackage{url}
\usepackage{adjustbox}
\usepackage{subcaption}
\usepackage{verbatim}
\usepackage{listings}
\usepackage{cite}
\usepackage{breakurl}  % Added for better URL handling

\title{Snowflake Naive Bayes and TPC-H}
\author{jarotek aff}
\date{December 2024}

\begin{document}

\maketitle

\section{Introduction}
Snowflake\cite{snowflake_2016} is a cloud native data platform blabla. 
In the following we wish to investigate and performance test two implementations of Naive Bayes on Yelp Reviews, a pure SQL implementation and a UDTF. We will discuss performance and ease of implementation, as well as under-the-hood details. 

\bigskip
Additionally we will conduct a standard TPC-H performance experiment with different configurations of Snowflake. 

\section{Methodology}

\subsection{Naive Bayes}
The LATERAL FLATTEN function does a few key things:

Takes a semi-structured column (like an array or object) and creates a new row for each element
Preserves the original row's context/data while expanding the nested elements
Returns a virtual table with helpful metadata columns (value, seq, key, path, index)

\subsection{TPC-H}
For my TPC-H implementation I use data from the Snowflake-supplied database \texttt{SNOWFLAKE\_SAMPLE\_DATA}. I test TPC-H query 1, 5 and 18, which I copied from a website\footnote{\url{https://docs.deistercloud.com/content/Databases.30/TPCH\%20Benchmark.90/Sample\%20querys.20.xml?embedded=true\#7b15827ccb57cbcec68be7a26953590c}} and modified them to use Snowflake date functions. 

\medskip \noindent Snowflake keeps a query history which, as everything else in Snowflake, can be queried on with SQL\footnote{\url{https://docs.snowflake.com/en/user-guide/performance-query-exploring#query-track-the-average-performance-of-a-query-over-time}}. Every query gets hashed, and the query hash can subsequently be used to query the query history and analyze the performance of every execution of the query.

\medskip \noindent I have a Python script under my Github repository (\url{tpch/snowflake_tpch_generate.py}), which takes as program argument the number of iterations. It then generates a SQL script that can be copied into a SQL Worksheet in the snowflake client and executed. 

\medskip \noindent The generation script puts a no-cache statement (\texttt{ALTER SESSION SET USE\_CACHED\_RESULT=FALSE;}) before each query. It then executes query 1, 5 and 18 on 4 sizes (extra small, small, medium, large) of warehouses, over scalefactor 1, 10, 100 and 1000. The warehouse is changed with the \texttt{USE WAREHOUSE} statement, and the scalefactors are changed with the \texttt{use schema} statement, (e.g. \texttt{use schema snowflake\_sample\_data.tpch\_sf1000}).

\medskip \noindent To distinguish between the 3 queries on the different scale factor and warehouse settings, I augment the queries to select a string literal to force a unique query hash, e.g. $$ \texttt{SELECT 'BISON\_WH\_XS\_Q1\_SF1000', (...)} $$

\medskip \noindent I have 48 distinct queries (3 queries * 4 warehouses * 4 scalefactors). I collect the query hashes manually from the query history, and I can then analyse the execution time with: 

\begin{verbatim}
    SELECT 
    query_parameterized_hash,
    COUNT(*) as query_count,
    AVG(total_elapsed_time) as mean_elapsed_time,
    STDDEV(total_elapsed_time) as std_elapsed_time,
    MAX(total_elapsed_time) as max_elapsed_time,
    MIN(total_elapsed_time) as min_elapsed_time,
    AVG(bytes_scanned)
FROM SNOWFLAKE.ACCOUNT_USAGE.QUERY_HISTORY
WHERE query_parameterized_hash IN ( ... )
AND DATE_TRUNC('day', start_time) >= CURRENT_DATE() - 30
GROUP BY query_parameterized_hash
ORDER BY mean_elapsed_time DESC;
\end{verbatim}

\noindent I execute the queries 20 times across a whole day. 


\section{Results and Discussion}

\FloatBarrier

\subsection{TPC-H}

I start my analysis by investigating correlation between megabytes scanned and execution time.

\begin{table}[htbp]
    \centering
    \begin{tabular}{lcc}
        \toprule
        \textbf{Warehouse} & \textbf{Query} & \textbf{Correlation} \\
        \midrule
        BISON\_WH\_L & Q1 & 0.999 \\
        BISON\_WH\_L & Q18 & 1.000 \\
        BISON\_WH\_L & Q5 & 1.000 \\
        \midrule
        BISON\_WH\_M & Q1 & 1.000 \\
        BISON\_WH\_M & Q18 & 1.000 \\
        BISON\_WH\_M & Q5 & 1.000 \\
        \midrule
        BISON\_WH\_S & Q1 & 1.000 \\
        BISON\_WH\_S & Q18 & 1.000 \\
        BISON\_WH\_S & Q5 & 1.000 \\
        \midrule
        BISON\_WH\_XS & Q1 & 1.000 \\
        BISON\_WH\_XS & Q18 & 0.999 \\
        BISON\_WH\_XS & Q5 & 1.000 \\
        \bottomrule
    \end{tabular}
    \caption{Pearson correlation between MB scanned and execution time by warehouse and query type (rounded to three decimals)}
    \label{tab:tpch_mbscanned_executiontime_correlations}
\end{table}

In table~\ref{tab:tpch_mbscanned_executiontime_correlations} we see that there is a definitive positive correlation between MB scanned and execution time. Given that we group by warehouse and query type, we end up checking correlation between the different scale factors and with a factor of 10 between each scale factor it is perhaps not so strange that the MB scanned end up having a strong correlation. 

\medskip \noindent 

\begin{figure}[htbp]
    \centering
    \includegraphics[width=1\textwidth]{imgs/tpch/query_scaling_comparison.png}
    \caption{Log log plots of the queries under different scale factors for each warehouse size. }
    \label{fig:query_scaling_comp}
\end{figure}

In figure~\ref{fig:query_scaling_comp} we see that query 18 scales the most linearly, whereas the other are penalized less by scale factor. 

\FloatBarrier

\subsection{Naive Bayes}

\subsubsection{SQL}
Overall Accuracy: 38.54\%

\section{Conclusion}

\newpage
\bibliographystyle{plain}
\bibliography{lit}

\end{document}